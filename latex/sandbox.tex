% Name: Demo Sandbox
% Description: Multiple fancy examples
% 
% Last modified
% Date: 2018/03/10
% Author: Jan Vorisek <jan@vorisek.me>

\documentclass[]{report}

\usepackage{hanzibox} % Display character boxes with various options
\usepackage{hyperref} % for http link
\usepackage{textcomp} % for text symbols - not required
\usepackage{lastpage} % show page/lastPage
\usepackage{fancyhdr}
\usepackage[top=3.5cm,bottom=2.5cm,left=3.5cm,right=2.5cm,headsep=1cm,footskip=0pt]{geometry}

\pagestyle{fancy}

% Header
\lhead{Sandbox}
\rhead{\textit{Jane Doe}}

% Footer
\lfoot{center of the footer!}
\cfoot{\thepage/\pageref{LastPage}}
\rfoot{\url{https://hanzisheets.vorisek.me}}

% Border under header and over footer
\renewcommand{\headrulewidth}{0.4pt}
\renewcommand{\footrulewidth}{0.4pt}

\begin{document}	
	\large\textbf{Explain words}
	\vspace{0.25cm}
	
	\hanzibox{pinyin=ni3, character=你, translation=you}{}
	\hanzibox{character=+}{inner=none, border=no}
	\hanzibox{pinyin=hao3, character=好, translation=good}{}
	\hanzibox{character={=}}{inner=none, border=no}
	\hanzibox{pinyin=ni3, character=你, translation=hello}{}\hspace{-0.4pt}%
	\hanzibox{pinyin=hao3, character=好, translation=}{}
	
	\large\textbf{Character boxes filled}
	\vspace{0.25cm}
	
	\begin{tabular}{@{}l@{}}
		\hanzibox{pinyin=wo3, character=容}{inner=none}\hspace{-0.4pt}%  
		\hanzibox{pinyin=wo3, character=容}{inner=none}\hspace{-0.4pt}%  
		\hanzibox{pinyin=wo3, character=容}{inner=none}\hspace{-0.4pt}%  
		\hanzibox{pinyin=wo3, character=容}{inner=none}\hspace{-0.4pt}%  
		\hanzibox{pinyin=wo3, character=容}{inner=none}\hspace{-0.4pt}%  
		\hanzibox{pinyin=wo3, character=容}{inner=none}\hspace{-0.4pt}%  
		\hanzibox{pinyin=wo3, character=容}{inner=none}\hspace{-0.4pt}%  
		\hanzibox{pinyin=wo3, character=容}{inner=none}\hspace{-0.4pt}%  
		\hanzibox{pinyin=wo3, character=容}{inner=none}\hspace{-0.4pt}%  
		\\
	\end{tabular}
	
	\large\textbf{Character boxes with \textit{inner cross}}
	\vspace{0.25cm}
	
	\begin{tabular}{@{}l@{}}
		\hanzibox{pinyin=wo3}{inner=cross}\hspace{-0.4pt}%  
		\hanzibox{pinyin=wo3}{inner=cross}\hspace{-0.4pt}%  
		\hanzibox{pinyin=wo3}{inner=cross}\hspace{-0.4pt}%  
		\hanzibox{pinyin=wo3}{inner=cross}\hspace{-0.4pt}%  
		\hanzibox{pinyin=wo3}{inner=cross}\hspace{-0.4pt}%  
		\hanzibox{pinyin=wo3}{inner=cross}\hspace{-0.4pt}%  
		\hanzibox{pinyin=wo3}{inner=cross}\hspace{-0.4pt}%  
		\hanzibox{pinyin=wo3}{inner=cross}\hspace{-0.4pt}%  
		\hanzibox{pinyin=wo3}{inner=cross}\hspace{-0.4pt}%  
		\\
	\end{tabular}
	
	\large\textbf{Character boxes with \textit{inner star} \, [\textbf{default}]}
	
	\begin{tabular}{@{}l@{}}
		\hanzibox{pinyin=wo3}{}\hspace{-0.4pt}%  
		\hanzibox{}{}\hspace{-0.4pt}%                      
		\hanzibox{}{}\hspace{-0.4pt}%                      
		\hanzibox{}{}\hspace{-0.4pt}%                      
		\hanzibox{}{}\hspace{-0.4pt}%                      
		\hanzibox{}{}\hspace{-0.4pt}%                      
		\hanzibox{}{}\hspace{-0.4pt}%                      
		\hanzibox{}{}\hspace{-0.4pt}%                      
		\hanzibox{}{}\hspace{-0.4pt}%                      
		\\
	\end{tabular}
	
	\large\textbf{Dialog}
	\vspace{0.25cm}
	
	\begin{tabular}{|c|l|}
		\hline%
		\rule{0pt}{65pt} % for nice top padding 
		\hanzidialog{Liang \pinyin{Lao2shi4}}{梁老师}{Učitelka Liang}%
		&   
		\hanzibox{pinyin=Lao3shi1, character=容, translation=trains}{}\hspace{-0.4pt}%
		\hanzibox{pinyin=wo3, character=容, translation=trains}{}\hspace{-0.4pt}%
		\hanzibox{pinyin=ma1, character=容, translation=train|s}{}\hspace{-0.4pt}%
		\hanzibox{pinyin=wo3, character=容, translation=trains}{}\hspace{-0.4pt}%
		\hanzibox{pinyin=wo3, character=容, translation=trains}{}\hspace{-0.4pt}%
		\hanzibox{pinyin=wo3, character=容, translation=trains}{}\hspace{-0.4pt}%
		\hanzibox{pinyin=wo3, character=容, translation=trains}{}\hspace{-0.4pt}%
		\hanzibox{pinyin=wo3, character=容, translation=trains}{}\hspace{-0.4pt}%
		\hanzibox{pinyin=wo3, character=容, translation=trains}{}\hspace{-0.4pt}%
		\\\hline%
		\rule{0pt}{65pt} % for nice top padding
		\hanzidialog{Wang Lang}{容容容}{Some lady}%
		&   
		\hanzibox{pinyin=Lao3, character=老, translation=starý}{}\hspace{-0.4pt}%
		\hanzibox{pinyin=shi1, character=师, translation=učitel}{inner=cross}\hspace{1cm}%
		\hanzibox{pinyin=ma1, translation=none}{inner=none}\hspace{-0.4pt}%
		\hanzibox{pinyin=ma1, translation=none}{inner=none}\hspace{1cm}%
		\hanzibox{pinyin=wo3, character=容, translation=trains}{}\hspace{-0.4pt}%
		\hanzibox{pinyin=wo3, character=容, translation=trains}{}\hspace{-0.4pt}%
		\\\hline
		\rule{0pt}{65pt} % for nice top padding 
		\hanzidialog{Wang Laoshi}{容容}{Paní Liang}%
		&   
		\hanzibox{pinyin=Lao3shi1, character=容, translation=trains}{inner=none}\hspace{-0.4pt}%
		\hanzibox{pinyin=wo3, character=容, translation=trains}{inner=none}\hspace{-0.4pt}%
		\hanzibox{pinyin=ma1, character=容, translation=train|s}{inner=cross}\hspace{-0.4pt}%
		\hanzibox{pinyin=wo3, character=容, translation=trains}{}\hspace{-0.4pt}%
		\hanzibox{pinyin=wo3, character=容, translation=trains}{}\hspace{-0.4pt}%
		\hanzibox{pinyin=wo3, character=容, translation=trains}{}\hspace{-0.4pt}%
		\hanzibox{pinyin=wo3, character=容, translation=trains}{}\hspace{-0.4pt}%
		\\\hline%
	\end{tabular}
\end{document} 